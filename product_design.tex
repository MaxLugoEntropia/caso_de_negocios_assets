\documentclass[11pt, a4paper]{article}
%\usepackage{geometry}
\usepackage[inner=1.5cm,outer=1.5cm,top=2.5cm,bottom=2.5cm]{geometry}
\pagestyle{empty}
\usepackage{graphicx}
\usepackage{fancyhdr, lastpage, bbding, pmboxdraw}
\usepackage[usenames,dvipsnames]{color}
\definecolor{darkblue}{rgb}{0,0,.6}
\definecolor{darkred}{rgb}{.7,0,0}
\definecolor{darkgreen}{rgb}{0,.6,0}
\definecolor{red}{rgb}{.98,0,0}
\usepackage[colorlinks,pagebackref,pdfusetitle,urlcolor=darkblue,citecolor=darkblue,linkcolor=darkred,bookmarksnumbered,plainpages=false]{hyperref}
\renewcommand{\thefootnote}{\fnsymbol{footnote}}

\pagestyle{fancyplain}
\fancyhf{}
%\lhead{ \fancyplain{}{Ciencia de datos aplicada II} }
%\chead{ \fancyplain{}{} }
%\rhead{ \fancyplain{}{Primavera 2025}}
%\rfoot{\fancyplain{}{page \thepage\ of \pageref{LastPage}}}
\fancyfoot[RO, LE] {page \thepage\ of \pageref{LastPage} }
\thispagestyle{plain}

%%%%%%%%%%%% LISTING %%%
\usepackage{listings}
\usepackage{caption}
\DeclareCaptionFont{white}{\color{white}}
\DeclareCaptionFormat{listing}{\colorbox{gray}{\parbox{\textwidth}{#1#2#3}}}
\captionsetup[lstlisting]{format=listing,labelfont=white,textfont=white}
\usepackage{verbatim} % used to display code
\usepackage{fancyvrb}
\usepackage{acronym}
\usepackage{amsthm}
\VerbatimFootnotes % Required, otherwise verbatim does not work in footnotes!



\definecolor{OliveGreen}{cmyk}{0.64,0,0.95,0.40}
\definecolor{CadetBlue}{cmyk}{0.62,0.57,0.23,0}
\definecolor{lightlightgray}{gray}{0.93}



\lstset{
%language=bash,                          % Code langugage
basicstyle=\ttfamily,                   % Code font, Examples: \footnotesize, \ttfamily
keywordstyle=\color{OliveGreen},        % Keywords font ('*' = uppercase)
commentstyle=\color{gray},              % Comments font
numbers=left,                           % Line nums position
numberstyle=\tiny,                      % Line-numbers fonts
stepnumber=1,                           % Step between two line-numbers
numbersep=5pt,                          % How far are line-numbers from code
backgroundcolor=\color{lightlightgray}, % Choose background color
frame=none,                             % A frame around the code
tabsize=2,                              % Default tab size
captionpos=t,                           % Caption-position = bottom
breaklines=true,                        % Automatic line breaking?
breakatwhitespace=false,                % Automatic breaks only at whitespace?
showspaces=false,                       % Dont make spaces visible
showtabs=false,                         % Dont make tabls visible
columns=flexible,                       % Column format
morekeywords={__global__, __device__},  % CUDA specific keywords
}

%%%%%%%%%%%%%%%%%%%%%%%%%%%%%%%%%%%%
\begin{document}

\section*{Ejercicio: Diseño de Producto}

El objetivo de este ejercicio es que seas capaz de diseñar un producto desde cero, partiendo de un problema real hasta llegar a la definición de su mercado objetivo y modelo de negocio.
A lo largo del ejercicio se evaluará la claridad del problema, la propuesta de valor, la coherencia de la solución y el entendimiento del mercado.

\vspace{0.5cm}

%------------------------------------------------
%------------------------------------------------

\section*{1. Definición de la empresa}

Antes de definir el problema y el producto, es necesario establecer el contexto general del proyecto y la identidad de la empresa que lo desarrollará.

\textbf{Nombre de la empresa}

Define el nombre de la empresa que desarrollará el producto.  
El nombre puede ser provisional y cambiar más adelante.

\textbf{Entrega esperada:}
\begin{itemize}
    \item Nombre de la empresa.
\end{itemize}

\vspace{0.3cm}

\textbf{Misión}

La misión describe el propósito actual de la empresa: qué hace, para quién lo hace y cómo crea valor.

Reflexiona:
\begin{itemize}
    \item ¿Cuál es la razón de existir de la empresa?
    \item ¿Qué problema general busca resolver?
    \item ¿A qué tipo de clientes sirve?
\end{itemize}

\textbf{Entrega esperada:}
\begin{itemize}
    \item Declaración de misión.
\end{itemize}

\vspace{0.3cm}

\textbf{Visión}

La visión describe el futuro deseado de la empresa y hacia dónde quiere llegar.

Reflexiona:
\begin{itemize}
    \item ¿Cómo debería verse la empresa en 5 a 10 años?
    \item ¿Qué impacto quiere tener en el mercado o la sociedad?
    \item ¿Qué la diferenciará de otras empresas en el futuro?
\end{itemize}

\textbf{Entrega esperada:}
\begin{itemize}
    \item Declaración de visión.
\end{itemize}

\vspace{0.3cm}

\textbf{Oportunidad de mercado}

Describe brevemente la oportunidad que existe en el mercado y por qué es relevante.

Reflexiona:
\begin{itemize}
    \item ¿Qué necesidad no está siendo bien atendida?
    \item ¿Por qué este es un buen momento para atacar este problema?
    \item ¿Qué tendencia, cambio tecnológico o contexto habilita esta oportunidad?
\end{itemize}

\textbf{Entrega esperada:}
\begin{itemize}
    \item Descripción clara de la oportunidad de mercado.
\end{itemize}

\vspace{0.3cm}



\section*{2. Definición del problema}

Identifica un problema real que valga la pena resolver.

Reflexiona sobre las siguientes preguntas:
\begin{itemize}
    \item ¿Qué tan doloroso o relevante es este problema para quien lo sufre?
    \item ¿Con qué frecuencia ocurre?
    \item ¿Las personas estarían dispuestas a pagar por resolverlo?
    \item ¿El valor que genera la solución es inmediato y tangible?
\end{itemize}

\textbf{Entrega esperada:}
\begin{itemize}
    \item Descripción clara y concisa del problema a resolver.
\end{itemize}

\vspace{0.3cm}

%------------------------------------------------

\section*{3. Identificación de los usuarios}

Analiza quiénes son las personas involucradas en el problema y su relación con la compra del producto.

Reflexiona:
\begin{itemize}
    \item ¿Quién sufre directamente el problema?
    \item ¿Es la misma persona quien paga por la solución?
    \item ¿Qué características demográficas, sociales o profesionales tienen?
\end{itemize}

\textbf{Entrega esperada:}
\begin{itemize}
    \item \textbf{User Persona}: definición con sus características del  usuario final que experimenta el problema.
    \item \textbf{Buyer Persona}: persona o entidad que toma la decisión de compra (puede o no ser el user-persona).
\end{itemize}


\vspace{0.3cm}

%------------------------------------------------

\section*{4. Definición de la solución}

Propón una solución concreta al problema definido.

Reflexiona:
\begin{itemize}
    \item ¿Cuál es tu producto o servicio?
    \item ¿Cómo resuelve el problema identificado?
    \item ¿En qué escenarios podría no funcionar?
    \item ¿Qué supuestos estás haciendo?
\end{itemize}

\textbf{Entrega esperada:}
\begin{itemize}
    \item Definición del producto.
    \item Lista de características principales.
\end{itemize}

\vspace{0.3cm}

%------------------------------------------------

\section*{5. Diferenciadores de mercado}

Propón una solución concreta al problema definido.

Reflexiona:
\begin{itemize}
    \item \textbf{¿Cuál es el beneficio de comprar conmigo en vez de mis competidores?}
    \item \textbf{¿Cuál es su diferenciador de producto?}   
    \item \textbf{¿qué hago diferente a otros o que me hace especial?}         
\end{itemize}

\textbf{Entrega esperada:}
\begin{itemize}
    \item Definición de diferenciadores
\end{itemize}

\vspace{0.3cm}


%------------------------------------------------

\section*{6. User Journey}

Describe el recorrido completo del usuario desde que identifica el problema hasta que obtiene valor del producto.

Reflexiona:
\begin{itemize}
    \item ¿Cómo descubre el usuario tu producto?
    \item ¿Cómo lo prueba o adopta?
    \item ¿Cómo interactúa con él?
    \item ¿En qué punto obtiene valor?
\end{itemize}

\textbf{Entrega esperada:}
\begin{itemize}
    \item Descripción paso a paso del user journey.
\end{itemize}

\vspace{0.3cm}

%------------------------------------------------

\section*{7. Modelo de negocio}

Define cómo el producto genera y captura valor.

\textbf{Entrega esperada:}
\begin{itemize}
    \item Business Model Canvas que incluya:
    \begin{itemize}
        \item Propuesta de valor
        \item Segmentos de clientes
        \item Canales
        \item Relación con clientes
        \item Fuentes de ingresos
        \item Recursos clave
        \item Actividades clave
        \item Socios clave
        \item Estructura de costos
    \end{itemize}
\end{itemize}

\vspace{0.3cm}

%------------------------------------------------

\section*{8. Definición del mercado}

A continuación se describen los conceptos utilizados para estimar el tamaño del mercado de un producto.

\begin{itemize}

    \item \textbf{TAM (Total Addressable Market)}\\
    Representa el tamaño total del mercado si el producto lograra capturar el \textbf{100\% de la demanda existente}, sin considerar limitaciones geográficas, operativas o de competencia.\\
    Responde a la pregunta: \textit{¿Cuál es el valor total del mercado si todos los posibles clientes compraran mi producto?}

    \vspace{0.2cm}

    \item \textbf{SAM (Serviceable Available Market)}\\
    Es la porción del TAM que el producto \textbf{puede atender realmente}, considerando restricciones como ubicación geográfica, idioma, regulaciones o alcance del producto.\\
    Responde a la pregunta: \textit{¿Qué parte del mercado total puedo servir con mi solución actual?}

    \vspace{0.2cm}

    \item \textbf{SOM (Serviceable Obtainable Market)}\\
    Es la fracción del SAM que la empresa \textbf{puede capturar de manera realista} en el corto o mediano plazo, considerando competencia, capacidad operativa, marketing y ventas.\\
    Responde a la pregunta: \textit{¿Qué porcentaje del mercado puedo conquistar de forma realista?}

\end{itemize}

\vspace{0.3cm}

\section*{9. Modelo financiero}

El objetivo de esta sección es construir un \textbf{modelo financiero en Excel} que permita entender la viabilidad económica del producto a lo largo del tiempo.

El modelo debe mostrar de forma clara:
\begin{itemize}
    \item Cuándo se realizan las inversiones iniciales y recurrentes.
    \item En qué periodos ocurren los costos operativos.
    \item A partir de qué momento se generan ingresos.
    \item Cómo evolucionan ingresos, costos y utilidad en el tiempo.
\end{itemize}

\textbf{Lineamientos del modelo:}
\begin{itemize}
    \item El modelo debe estar organizado por periodos (mensual o anual).
    \item Debe incluir supuestos explícitos (precios, crecimiento, costos, etc.).
    \item Los supuestos deben estar claramente separados de los cálculos.
\end{itemize}

\textbf{Componentes mínimos del modelo:}
\begin{itemize}
    \item Inversión inicial (desarrollo, infraestructura, marketing, etc.).
    \item Costos fijos y variables.
    \item Proyección de ingresos.
    \item Flujo de caja.
    \item Punto de equilibrio (break-even).
\end{itemize}

\textbf{Entrega esperada:}
\begin{itemize}
    \item Archivo en Excel con el modelo financiero.
    \item Breve explicación de los supuestos utilizados.
\end{itemize}

El objetivo no es la precisión absoluta, sino demostrar comprensión sobre \textbf{cómo y cuándo el negocio crea valor económico}.


%------------------------------------------------

\section*{10. Identidad del producto: nombre y logotipo}

El objetivo de esta sección es definir la \textbf{identidad visual y conceptual del producto}, de manera que sea coherente con el problema, el usuario y la propuesta de valor.

\textbf{Selección del nombre del producto}

Reflexiona sobre las siguientes preguntas:
\begin{itemize}
    \item ¿El nombre comunica el valor del producto?
    \item ¿Es fácil de recordar y pronunciar?
    \item ¿Es coherente con el mercado y el tipo de usuario?
    \item ¿Transmite confianza, innovación o el atributo principal del producto?
\end{itemize}

\textbf{Entrega esperada:}
\begin{itemize}
    \item Nombre del producto.
    \item Breve justificación del nombre elegido.
\end{itemize}

\vspace{0.3cm}

\textbf{Diseño del logotipo}

El logotipo debe representar visualmente la esencia del producto.

Reflexiona:
\begin{itemize}
    \item ¿Qué emociones o valores debe transmitir?
    \item ¿Dónde se utilizará el logotipo (web, app, redes, presentaciones)?
    \item ¿Funciona tanto en color como en blanco y negro?
\end{itemize}

\textbf{Entrega esperada:}
\begin{itemize}
    \item Propuesta de logotipo (imagen o boceto).
    \item Paleta de colores y tipografía (opcional).
\end{itemize}

El objetivo no es lograr un diseño perfecto, sino construir una identidad \textbf{clara, coherente y alineada con el producto}.

\vspace{0.3cm}

La siguiente liga encontrarás una presentación donde podrás poner la información sobre tu producto para crear un pitch deck de negocios.

\href{https://gamma.app/docs/Diseno-de-producto-3qtwj5t7jasj6bm}{Presentación: Diseño de Producto}

\end{document}



